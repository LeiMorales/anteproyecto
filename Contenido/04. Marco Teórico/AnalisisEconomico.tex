\subsection*{Análisis económico}
\addcontentsline{toc}{subsection}{Análisis económico}

El contexto macroeconómico de Colombia (y en particular de Bogotá) en el periodo 2024–2025 se caracteriza por un crecimiento moderado, una inflación en descenso y un mercado laboral en recuperación. De acuerdo con proyecciones sectoriales, el \textbf{PIB nacional} presenta un crecimiento estimado de entre \textit{1,8\% y 2,1\% anual} \parencite[p. 3]{Asobancaria01}. En Bogotá D.C., motor económico del país, el PIB aumentó un \textbf{2,8\%} al cierre de 2024, consolidando una recuperación que exige modelos de negocio escalables e integrados al sistema financiero formal \parencite[p. 12]{RevistaCIES01}. Los sectores con mayor impulso han sido comercio al por mayor/minorista, transporte y alojamiento; no obstante, el crecimiento del sector tecnológico depende de una regulación adaptativa que facilite la colaboración entre las \textit{Fintech} y la banca tradicional, permitiendo que la innovación técnica se traduzca en servicios de mayor calidad y accesibilidad para el consumidor \parencite[p. 15]{RevistaCIES01}.

En materia de \textbf{inflación}, tras el pico de 2023, el país experimentó una moderación significativa; el año 2024 cerró con una inflación anual de \textit{5,20\%} \parencite[p. 5]{Asobancaria01}. A mediados de 2025 la inflación anual rondaba el \textit{4,8–5,0\%}, acercándose al rango meta del Banco Central (2–4\%). En respuesta, la tasa de interés de política se redujo a \textbf{10,25\%} en septiembre de 2024, mejorando las condiciones para servicios financieros que requieren fondeo y captación \parencite[p. 2]{Asobancaria01}. Este entorno de menores tasas incentiva la inversión en innovación y desarrollo de nuevas tecnologías financieras, promoviendo una competencia saludable que diversifica la oferta para el consumidor final \parencite[p. 17]{RevistaCIES01}.

El \textbf{mercado laboral} muestra una tendencia positiva, especialmente en la capital. Bogotá registró una tasa de desocupación del \textbf{9,0\%} en el primer semestre de 2025, situándose por debajo del promedio de la mayoría de regiones del país y mostrando una mejora frente al 10,4\% del año anterior \parencite[p. 1, 3]{DANE_Regiones}. Asimismo, la ciudad presenta la Tasa de Ocupación (TO) más alta del territorio nacional con un \textbf{62,1\%} \parencite[p. 4]{DANE_Regiones}. Sin embargo, la informalidad persiste como un reto estructural; a nivel nacional, la proporción de ocupados informales fue del \textbf{55,4\%} para el trimestre septiembre-noviembre de 2025 \parencite[p. 4]{DANE01}, aunque en Bogotá este índice es notablemente inferior, situándose en el \textbf{32,4\%} \parencite[p. 11]{DANE01}.

Para nuestro servicio, la situación de la juventud es crítica: la tasa de desocupación nacional para personas entre 15 y 28 años se ubicó en el \textbf{16,1\%} para el trimestre agosto-octubre de 2025 \parencite[p. 1]{DANE_Juventud2025}. Este contexto de alta vulnerabilidad laboral en jóvenes y un gran segmento informal (que carece de contabilidad separada para sus gastos \parencite[p. 16]{DANE01}) sustenta la necesidad de herramientas de gestión de dinero y educación financiera digna \parencite[p. 20]{RevistaCIES01}. El salario mínimo aumentó 9,5\% para 2025, lo cual, si bien mejora el poder adquisitivo, podría presionar los costos de los servicios financieros \parencite[p. 4]{Departamento01}.

%Pendiente por incluire referencias%
%Anadir referencia para la pobreza monetaria, acceso de hogares a internet%

En cuanto a \textbf{pobreza y acceso}, la pobreza monetaria alcanzó el \textit{31,8\%} en 2024, un minimo histórico, indicando que casi dos tercios de la población supera la línea de pobreza. La inclusión financiera es alta, con un \textbf{95–96\%} de adultos con al menos un producto financiero \parencite[p. 2]{BancadeOportunidades01}. Sin embargo, esta inclusión es nominal y no siempre se traduce en bienestar, la conectividad digital ha mejorado, pero
todavía solo el 65,6 \% de los hogares tenía acceso a internet en 2024. En Bogotá la cifra es mayor, pero la brecha rural-urbana persiste. Esto implica que un plan de negocio basado en tecnología
requiere estrategias para educación y soporte offline/in-persona en poblaciones vulnerables.

%PArrafo pendiente por verificar referencias y contenido%
Finalmente, las \textbf{condiciones del sistema financiero local} son dinámicas. Se observa un fuerte impulso a la digitalización (con iniciativas de \textit{open finance} y Banca Abierta aprobadas en 2022), lo que abre oportunidades para nuevos modelos de negocio basados en datos. Por ejemplo, Nequi reporta funcionalidades avanzadas (análisis de gastos, metas de ahorro, “Resumen Nequi”) para sus usuarios, aprovechando el ecosistema digital. No obstante, la alta concentración bancaria (los cinco bancos más grandes controlan más de 75\% de los activos financieros) puede encarecer el crédito a mediano plazo. A nivel regulatorio, la Superintendencia Financiera exige sistemas de prevención de lavado de activos (SARLAFT) y ciberseguridad, lo que se traduce en costos iniciales para emprendimientos financieros.

%Parrafo pendiente por verificar contenido%
En resumen, el \textbf{entorno económico colombiano para 2024–2025} presenta un moderado crecimiento con inflación en descenso, descenso del desempleo y mejora en las condiciones de ingreso real. Las perspectivas macroeconómicas –PIB proyectado ~1,9\% (2024) y 2,9\% (2025) según BanRep apuntan a un ambiente estable pero competitivo. Para nuestro servicio, esto significa un mercado con \textbf{potencial de demanda} (pues la población tiene mayor liquidez disponible y alto índice de bancarización), pero también con consumidores exigentes y costos operativos (tasa de interés de mercado, inflación) que deben considerarse en la estructura financiera del plan.
\section*{Modelo de Negocio}
\addcontentsline{toc}{section}{Modelo de Negocio}

Para la fase de desarrollo de nuestro modelo de negocio, llevamos a cabo la elección de un patrón para distinguir entre los distintos grupos de clientes a los que se dirige nuestro proyecto. Para ello, seguimos el marco propuesto por Alexander Osterwalder y Yves Pigneur, autores del libro ``\textit{Generación de modelos de negocio}'' \parencite{BusinessModel01}. Se decidió elegir esta opción, ya que nos permitió visualizar y gestionar las relaciones complejas existentes en nuestro entorno empresarial.

\begin{figure}[H]
	\centering
	\includegraphics[width=15cm]{Imagenes/ModeloCanvas.png}
	\caption[{Modelo CANVAS.}]{\centering Modelo CANVAS. \textit{Fuente:} Autores.} 
	\label{fig:modelocanvasproy}
\end{figure}

La Figura \ref{fig:modelocanvasproy} presenta una visualización concisa de la estructura de nuestro modelo de negocio. Elaborada mediante el Modelo Canvas previamente definido, en ella se establece el propósito principal del proyecto a través de una breve descripción, la forma en que se financiará, el tipo de cliente al que se dirige el negocio y la propuesta de valor que se ofrece al mercado. Esta figura proporciona una guía sólida para la toma de decisiones estratégicas y favorece una comunicación más efectiva del enfoque empresarial. Esto constituye un paso fundamental para comprender y avanzar en la implementación del proyecto.

\input{Contenido/06. Modelo de Negocio/SegmentoDelMercado}

\input{Contenido/06. Modelo de Negocio/PropuestaDeValor}

\subsection*{Canales de Distribución (CD)}
\addcontentsline{toc}{subsection}{Canales de Distribución (CD)}
Es la forma en la que especificamos como será el contacto con los clientes, para ello, el modelo de distribución de nuestro proyecto será principalmente a través de un sitio web intuitivo. Junto a divulgación en redes sociales (haciendo presencia en plataformas donde se concentra el público objetivo como Facebook, TikTok, etc.) y marketing de contenidos para captación. Junto a ello, dentro de las posibilidades se contemplan alianzas con universidades, ONGs y fintech locales para difusión (talleres, seminarios), y por último publicidad en línea paga (Google Ads).

\subsection*{Relaciones con Clientes (CLI)}
\addcontentsline{toc}{subsection}{Relaciones con Clientes (CLI)}
Este apartado busca enfatizar lo fundamental de establecer relaciones personalizadas con cada uno de nuestros clientes como parte integral del proyecto. Para ello, dentro de nuestro proyecto buscaremos establecer un servicio mayormente automatizado y personalizado (con seguimiento basado en machine learning) combinado con una relación personal definida por opciones personalizadas (comunidades de usuario o servicio técnico) buscando prestar de forma proactiva la atención del cliente y reforzando su aprendizaje por retroalimentación positiva de forma gamificada (insignias, retos mensuales).

\subsection*{Fuentes de Ingresos (FI)}
\addcontentsline{toc}{subsection}{Fuentes de Ingresos (FI)}
Es importante definir las fuentes de ingresos para garantizar la sostenibilidad del proyecto. Para lograrlo, nuestro proyecto establecerá un modelo \textit{freemium}, por lo que existirán dos fuentes de ingresos, la primera será mediante ingresos por  publicidad dentro de la página en su versión gratuita, y la última será una suscripción premium la cual desbloquea funciones avanzadas (planificación detallada, asesoría personalizada) y sin anuncios de pago mensual. De forma opcional, también se consideran alianzas pagas con empresas y publicidad segmentada (e.g. ofertas de productos financieros adecuados al perfil del usuario).

\subsection*{Actividades Clave (AC)}
\addcontentsline{toc}{subsection}{Actividades Clave (AC)}
Se centra en describir las actividades importantes que se debe emprender dentro de nuestro proyecto, las cuales son el desarrollo y mantenimiento de una plataforma web intuitiva que permita a los usuarios entender sobre conceptos financieros mediante uso de contenido didáctico digital con ruta personalizada por machine learning, además de la promoción activa de la plataforma. Para enfatizar la inclusión, se monitorean métricas de uso en poblaciones vulnerables y los efectos que tiene la plataforma sobre este público.

\subsection*{Recursos Clave (RC)}
\addcontentsline{toc}{subsection}{Recursos Clave (RC)}
Esta sección detalla los insumos que se necesitan para el funcionamiento del negocio, estos recursos serán compuesto en gran medida por el capital humano, equipo que permite mantener la plataforma web, su infraestructura y el soporte técnico necesario. Además para el capital tecnológico se encontrarán herramientas de software y hardware que son necesarias para el mantenimiento de la plataforma.

\subsection*{Socios Clave (SC)}
\addcontentsline{toc}{subsection}{Socios Clave (SC)}
Se busca establecer la red de proveedores y socios estratégicos importante que colaborarán con el funcionamiento y visibilidad del modelo de negocios, en nuestro proyecto se consideran jóvenes entre 18 a 24, nómadas financieros y otras personas interesadas en temas financieros, así como posibles alianzas con instituciones educativas, asociaciones de trabajadores informales y entidades públicas (p. ej. programas de inclusión financiera).

\subsection*{Estructura de Costos(EC)}
\addcontentsline{toc}{subsection}{Estructura de Costos(EC)}
Se detallan los costos asociados al inicio de operaciones del modelo de negocio. Esto significa considerar de forma cuidadosa los recursos financieros necesarios así como los costos relacionados con el talento humano (encargado del mantenimiento de la plataforma), gastos destinados a publicidad y marketing para la promoción del proyecto. Además de los costos de servicios tecnológicos como la adquisición de hardware y software.
\subsection*{Justificación del Problema}
\addcontentsline{toc}{subsection}{Justificación del Problema}
El panorama de inclusión financiera en Colombia muestra avances cuantitativos importantes, como se dijo anteriormente, alrededor del 94.6\% de las personas adultas tiene al menos un producto financiero formal para el año 2023, siendo el producto más usual las cuentas de depósito con un 94\% de los adultos con al menos una cuenta de ahorros \parencite{BancadeOportunidades01}. Pero esa cobertura varía dramáticamente según el contexto socioeconómico y regional. En la cual, el uso de al menos un producto financiero es alto en las zonas urbanas principales (ciudades y municipios intermedios, incluyendo Bogotá) con porcentajes que rondan entre el 70\% al 90\%, mientras que en los municipios rurales apenas ronda el 55\% al 60\%, lo que implica una brecha urbano-rural de entre el 10\% al 30\%. \parencite[p.~18, 21-25]{BancadeOportunidades01}.

A su vez, las mujeres adultas tienen niveles de acceso inferiores a los hombres; por ejemplo, en 2024 el 92,5\% de las mujeres adultas usan o tienen acceso a productos financieros, lo cual, a pesar de aumentar la proporción de mujeres que acceden al sistema financiero, la brecha de género aumento, mientras que en 2018 la diferencia se estimaba en 4.6 puntos porcentuales, para el año 2024, esta alcanzó los 6.9 puntos \parencite[p.~2]{Asobancaria01}.

Estas disparidades se agravan por la alta informalidad laboral, de acorde con el DANE, para el  trimestre móvil entre septiembre y noviembre de 2025, el 55,4\% de los ocupados a nivel nacional estaban en la informalidad y ese porcentaje superaba el 83\% en las zonas rurales, mientras que en las 13 principales ciudades (entre ellas Bogotá) la informalidad alcanzó el 43,1\%. \parencite[p.~1]{DANE01}.

Por lo que, en la práctica, los trabajadores informales o con bajos ingresos, los cuales cuentan sin un acceso permanente a la seguridad social ni a canales formales de crédito tienen muchas limitaciones para abrir cuentas, acceder a microcréditos o seguros, lo que reduce aún más su inclusión financiera efectiva.

Por otra parte, la alfabetización financiera de la población colombiana es muy baja, lo que dificulta que los usuarios conozcan y utilicen bien los productos formales disponibles. Una investigación realizada dentro del banco de la república basado en una encuesta nacional encontró que, al presentar un cuestionario de preguntas sobre la tasa de interés, la inflación y la diversificación del riesgo en las grandes ciudades (Bogotá, Medellín, Cali, Bucaramanga y Barranquilla) en el año 2023. Halló que en temas de interés se ronda el 42.4\% de respuestas correctas (a través de realizar un promedio en todas las situaciones laborales) mientras que el 49.5\% contestaba correctamente temas de riesgos. Siendo que, por lo general, el 16.1\% de los encuestados respondieron exitosamente todas las preguntas. \parencite[p.~4]{Departamento01}.

Esta misma investigación demuestra que el nivel de educación financiera es bajo en jóvenes (el cual ronda el 18.2\% de las 3 respuestas correctas), en hogares de menores ingresos o nivel educativo (sin superar el 20\% de las 3 respuestas correctas en los primeros 4 estratos), y en personas dedicadas al trabajo informal, lo cual confirma que la falta de entendimiento de conceptos financieros es una barrera importante para la inclusión. No sorprende entonces que, según esa investigación, existe una ``urgencia de seguir desarrollando iniciativas para mejorar la educación económica y financiera en Colombia. Esta necesidad especialmente crítica en grupos vulnerables con bajos niveles de alfabetización financiera'' \parencite[p.~5-7]{Departamento01}. En otras palabras, mejorar la capacidad de ahorro y manejo del presupuesto personal es clave para que más personas se incorporen al sistema formal y aprovechen sus beneficios.

Ante este escenario, la implementación de un servicio que permita entender a sus usuarios de educación y planificación financiera de forma personalizada en Bogotá se presenta como una solución viable y necesaria. Al enfocarse en usuarios con bajo o nulo conocimiento financiero, dicho servicio podría explicarles de manera clara sobre la elaboración de presupuestos y un aprendizaje didáctico y entendible. Así se haría un esfuerzo activo para ayudar a aumentar los conocimientos financieros explicados, pues la evidencia muestra que se debe realizar un esfuerzo de coordinación entre sectores privados y públicos y expandir el alcance de programas de educación financiera \parencite[p.~7]{Departamento01}. Además, al apoyar a quienes trabajan en la economía informal a comprender los productos formales, se reducirían las barreras prácticas de acceso y se fomentaría la bancarización efectiva.
\section*{Planteamiento del Problema}
\addcontentsline{toc}{section}{Planteamiento del Problema}
La falta de alfabetización financiera limita la capacidad de los individuos para enfrentar la complejidad de productos bancarios, créditos e inversiones. Estudios nacionales resaltan esta carencia, debido a los bajos niveles de educación financiera en el país, evidenciados por las altas tasas de interés que pagan los hogares. El desconocimiento y desinformación generalizados sobre temas económicos y financieros impiden a la ciudadanía tomar decisiones fundamentadas, afectando negativamente su bienestar personal y familiar \parencite[p.~3]{Superintendencia01}. Ante este panorama, organismos oficiales como la Banca de las Oportunidades destacan la necesidad de elevar los conocimientos financieros de la población para que “tomen decisiones informadas y responsables” en materia de ahorro, gasto e inversión.

A nivel nacional, se estima que un total de 64\% de la población planifica para menos e un mes o no tiene planes financieros en lo absoluto, mientras que 58\% se le dificulta cubrir estos gastos \parencite[p.~9]{ComisionIntersectorial01}. A pesar de que Bogotá tiene un alto grado de sofisticación financiera, existen varios problemas de raíz, según la encuesta IEFIC \textit{(Encuesta de Carga Financiera y Educación Financiera de los Hogares)} se encontró que la mayoría de encuestados tenían signos de sufrir dificultados financieras inclusive en la población bancaria. Esto se manifiesta de forma principal en los sectores de ingreso más bajo y de menor educación \parencite[p.~15]{BancoMundial01}. Se manifiesta, además, que una de las razones para que este grupo de personas no puedan acceder a ciertos productos o instancias financieras es debido a la falta de conocimiento sobre como usarlos o de que tratan. \parencite[p.~24]{Inclusionfinan01}.

Además para el caso de jóvenes que tienen una edad entre 18 a 24 años. A pesar de tener muchos más conocimientos financieros a comparación de grupos con mayor edad. Les cuesta en gran medida el control y seguimiento de sus gastos, adicionalmente, también les cuesta la planificación de un presupuesto en comparación con otros grupos de edades. \parencite[p.~41-44]{BancoMundial01}

\begin{figure}[H]
	\centering
	\includegraphics[width=10cm]{Imagenes/ArboldeProblemas.png}
	
	\caption[{Árbol del problema}]{\centering Árbol del problema. \textit{Fuente:} Autores. Adaptación \parencite{Economipedia01}} 
	\label{fig:arboldeproblemas}
\end{figure}

\input{Contenido/02. Planteamiento/DescripcionDelProblema}

\subsection*{Formulación del Problema}
\addcontentsline{toc}{subsection}{Formulación del Problema}
¿Cómo puede una plataforma digital de asesoramiento financiero superar las barreras asociadas a la baja alfabetización financiera y la limitada personalización de contenidos, para brindar orientación accesible, confiable y adaptada a las necesidades de diversos perfiles de usuarios, promoviendo una mejora real en sus hábitos económicos y maximizando el impacto social y la sostenibilidad del modelo de negocio?

\subsection*{Justificación del Problema}
\addcontentsline{toc}{subsection}{Justificación del Problema}
El panorama de inclusión financiera en Colombia muestra avances cuantitativos significativos; al cierre de 2023, aproximadamente el 94,6\% de los adultos contaba con al menos un producto financiero formal, siendo las cuentas de depósito el instrumento predominante con una cobertura del 94\% \parencite{BancadeOportunidades01}. No obstante, esta cobertura presenta disparidades críticas según el contexto socioeconómico. Mientras que en zonas urbanas como Bogotá el uso de productos financieros oscila entre el 70\% y el 90\%, en los municipios rurales esta cifra desciende a rangos de entre 55\% y 60\%, evidenciando una brecha de acceso que alcanza hasta los 30 puntos porcentuales \parencite[pp.~18, 21-25]{BancadeOportunidades01}.

A esta brecha regional se suma una disparidad de género creciente. En 2024, aunque el 92,5\% de las mujeres adultas accedía al sistema, la brecha frente a los hombres se amplió de 4,6 puntos porcentuales en 2018 a 6,9 puntos en 2024 \parencite[p.~2]{Asobancaria01}. Estas cifras se ven agravadas por la alta informalidad laboral; de acuerdo con el DANE, para el trimestre móvil de septiembre a noviembre de 2025, la informalidad alcanzó el 55,4\% a nivel nacional, superando el 83\% en áreas rurales y situándose en un 43,1\% en las principales ciudades como Bogotá \parencite[p.~1]{DANE01}.

Por lo que, en la práctica, los trabajadores informales o con bajos ingresos, los cuales cuentan sin un acceso permanente a la seguridad social ni a canales formales de crédito tienen muchas limitaciones para abrir cuentas, acceder a microcréditos o seguros, lo que reduce aún más su inclusión financiera efectiva.

Por otra parte, la alfabetización financiera de la población colombiana es muy baja, lo que dificulta que los usuarios conozcan y utilicen bien los productos formales disponibles. Una investigación realizada dentro del banco de la república basado en una encuesta nacional encontró que, al presentar un cuestionario de preguntas sobre la tasa de interés, la inflación y la diversificación del riesgo en las grandes ciudades (Bogotá, Medellín, Cali, Bucaramanga y Barranquilla) en el año 2023. Halló que en temas de interés se ronda el 42.4\% de respuestas correctas (a través de realizar un promedio en todas las situaciones laborales) mientras que el 49.5\% contestaba correctamente temas de riesgos. Siendo que, por lo general, el 16.1\% de los encuestados respondieron exitosamente todas las preguntas. \parencite[p.~4]{Departamento01}.

Esta misma investigación demuestra que el nivel de educación financiera es bajo en jóvenes (el cual ronda el 18.2\% de las 3 respuestas correctas), en hogares de menores ingresos o nivel educativo (sin superar el 20\% de las 3 respuestas correctas en los primeros 4 estratos), y en personas dedicadas al trabajo informal, lo cual confirma que la falta de entendimiento de conceptos financieros es una barrera importante para la inclusión. No sorprende entonces que, según esa investigación, existe una ``urgencia de seguir desarrollando iniciativas para mejorar la educación económica y financiera en Colombia. Esta necesidad especialmente crítica en grupos vulnerables con bajos niveles de alfabetización financiera'' \parencite[p.~5-7]{Departamento01}. En otras palabras, mejorar la capacidad de ahorro y manejo del presupuesto personal es clave para que más personas se incorporen al sistema formal y aprovechen sus beneficios.

%Verifico este parrafo, yo creo que ahí plantea correctamente la perspectiva y lo justicica con el Machine Learning%
Desde la perspectiva de la \textbf{Ingeniería de Sistemas}, esta problemática justifica el desarrollo de una solución tecnológica que supere las limitaciones de los métodos tradicionales de enseñanza. La implementación de una plataforma digital no solo permite la escalabilidad del conocimiento, sino que facilita la personalización mediante algoritmos de \textit{machine learning} que adaptan los contenidos al flujo de caja y comportamiento real del usuario informal o joven. De este modo, se busca transformar el acceso nominal en una bancarización efectiva, reduciendo las barreras de entendimiento mediante herramientas interactivas que promuevan una cultura de ahorro estratégica y sostenible \parencite[p.~7]{Departamento01}.

Ante este escenario, la implementación de un servicio tecnologico que permita entender a sus usuarios de educación y planificación financiera de forma personalizada en Bogotá se presenta como una solución viable y necesaria. Al enfocarse en usuarios con bajo o nulo conocimiento financiero, dicho servicio podría explicarles de manera clara sobre la elaboración de presupuestos y un aprendizaje didáctico y entendible. Así se haría un esfuerzo activo para ayudar a aumentar los conocimientos financieros explicados, pues la evidencia muestra que se debe realizar un esfuerzo de coordinación entre sectores privados y públicos y expandir el alcance de programas de educación financiera \parencite[p.~7]{Departamento01}. Además, al apoyar a quienes trabajan en la economía informal a comprender los productos formales, se reducirían las barreras prácticas de acceso y se fomentaría la bancarización efectiva.
\section*{Introducción}
\phantomsection
\addcontentsline{toc}{section}{Introducción}
El avance de la tecnología ha logrado transformar de forma profunda el ámbito financiero a un nivel global. Logrando crear plataformas que conocemos como las empresas de tecnología financiera (mejor conocidas como \textit{Fintech}) las cuales surgen como actores principales para la innovación dentro del sector financiero, siendo su impacto notable para la inclusión financiera y optimización de los procesos operativos. Sin embargo, a nivel nacional, a pesar de este potencial que tienen las Fintech para mejorar la inclusión financiera, existen múltiples obstáculos que impiden este crecimiento. \parencite[p.~8]{RevistaCIES01}.

Una de las principales se encuentra en el manejo de la cultura de educación financiera presente en nuestro país, la cual se ubica en niveles bastante bajos, al no existir evaluaciones contantes de las obligaciones financieras que se toman, no se conoce en nuestro país varios términos tales como las tasas de interés, inflación y de presupuestos. Esto implica un serio problema al desconocer el valor del dinero y la carencia del verdadero valor del dinero \parencite[p.~13]{UniversidadLibre01}.

Para el caso de la ciudad de Bogotá, a pesar de poseer una educación financiera por encima del promedio no es suficiente. Según una investigación del Colegio de Estudios Superiores de Administración, más de la mitad de los jóvenes universitarios no anotan sus gastos, el 49,5\% no cuenta con un plan de ahorros y más del 70\% de estos jóvenes universitarios no invierten esos ahorros. Lo que refleja una falta de herramientas y conocimientos para poder monitorear y controlar sus finanzas. \parencite[p.~57]{CESA01}.

Esta propuesta tiene como base la premisa de buscar ayudar a mitigar aquellos desafíos que se presentan sobre el entendimiento de las bases financieras y ofrecer una alternativa para entender acerca del mundo de las finanzas, sus conceptos y para el manejo de las finanzas de cada uno de nuestros usuarios. Logrando de esta forma promover un mayor aprendizaje económico en nuestra ciudad y, por lo tanto, del país.

Si bien, aunque ya existen plataformas financieras y contenidos de educación financiera, la mayoría de estas herramientas buscan generar un registro de gastos o de cursos muy poco contextualizado, apoyándose en recomendaciones unidireccionales sin alterar el comportamiento del usuario. La plataforma propuesta se busca diferenciar buscando integrar a los jóvenes o personas que inicien su vida financiera con la integración de contenidos hipercontextualizados sobre situaciones típicas de este segmento (tales como créditos, tasas de interés, pagos de cuotas, etc). 

Para cumplir aquello, buscamos realizar motores de recomendaciones que personalizan aquellas rutas de aprendizaje y simulaciones dependiendo del perfil financiero utilizando machine learning y un sistema de seguimiento de hábitos que mide de forma objetiva la evolución del bienestar financiero (como el presupuesto, ahorro y endeudamiento) y ajusta automáticamente la experiencia. Por lo tanto, la plataforma no solo busca informar, sino que acompaña al usuario en un proceso iterativo de diagnóstico, entrenamiento y mejora tanto en sus hábitos financieros como en el conocimiento del mismo.
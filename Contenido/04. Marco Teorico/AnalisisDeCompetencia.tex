\subsection*{Análisis de competencia}
\addcontentsline{toc}{subsection}{Análisis de competencia}

El entorno competitivo de los servicios financieros y la educación financiera de forma digital en Colombia es bastante amplio y heterogéneo. Coexistiendo con actores tradicionales, las fintech y los neobancos, lo que configura una sobreoferta de soluciones que entre las cuales compiten por la atención de los mismos usuarios, por tanto, mencionaremos algunos de los principales actores de la competencia:

\begin{itemize}
	\item \textbf{Banca tradicional:} La banca tradicional (Bancolombia, BBVA, Davivienda, Banco de Bogotá, entre otros), concentran la mayor parte de los activos financieros del país. Además de contar con una extensa presencia física y canales digitales consolidados. Estas entidades ofrecen programas de educación financiera (seminarios, talleres, cápsulas, etc) y funcionalidades básicas dentro de sus aplicaciones (consultas o pagos); sin embargo, la asesoría personalizada está ligada más que todo hacia productos contratados y orientada hacia la comercialización del portafolio, más que un entrenamiento de habilidades financieras para personas principiantes. Por este motivo, la oferta que entrega es más generalista y poco adaptada para las necesidades de los jóvenes que inician su vida financiera o de usuarios sin experiencia previa en finanzas.
	
	\item \textbf{Fintech y neobancos:} El ecosistema fintech colombiano ha crecido de forma sostenida durante los últimos años. Casos como, por ejemplo, \textbf{Nequi} (de Grupo Bancolombia), \textbf{Daviplata} (de Davivienda) y muchos otros más han “revolucionado la manera en que los colombianos manejan su dinero”, logrando masificar el uso de cuentas digitales, pagos y microcréditos mediante interfaces simples. Mientras que estas soluciones logran incorporar elementos de planificación básica (tales como metas de ahorros, alertas de pago y demás) y en cierto modo campañas de educación financiera integradas; su foco principal siempre ha sido la adopción de productos financieros y el volumen de aquellas transacciones, sobre la construcción de un currículo de formación financiera o el seguimiento del aprendizaje del usuario. Aunque tienen presencial en grupos como los jóvenes y la población no bancarizada, la personalización pedagógica se limita a la segmentación comercial por sobre el aprendizaje financiero.
	
	\item \textbf{Aplicaciones educativas y consultorías digitales:} Existen apps específicamente diseñadas para la educación y gestión financiera personal, los cuales buscan combinar el registro de ingresos y gastos por medio de recursos didácticos, talleres y otros. Así como cursos en línea, blogs y contenido de entidades públicas. Estas soluciones aportan material pedagógico valioso, pero en muchos casos ofrecen contenidos estandarizados, sin tener en cuenta el contexto del usuario en sí y de situaciones más específicas, y sin un uso intensivo de modelos de usuarios o algoritmos que puedan ajustar dinámicamente la dificultad o la retroalimentación que aprecia el usuario.
	
	\item \textbf{Finfluencers y contenidos finacieros en redes sociales:} Se suman los finfluencers (una unión de las palabras de ``\textit{financiero}'' e ``\textit{influencer}'') y de contenidos de redes sociales y plataformas con videos cortos, altamente atractivos y en algunos casos usan herramientas de generación de inteligencia artificial para producir consejos y trucos de inversión. A pesar de que estos actores contribuyen a visibilizar la educación financiera entre este público, diversos análisis señalan que existe un riesgo en acentuarse en este formato: la sobresimplificación de mensajes, la ausencia de veracidad independiente, conflictos de interés y promoción de estrategias de alto riesgo sin el contexto adecuado son varios de los riesgos que se puede enfrentar. En la práctica, los usuarios solo reciben recomendaciones unidireccionales, sin un diagnóstico, sin sistema de evaluación de la lógica detrás de estas recomendaciones ni un acompañamiento adecuado, limitando el verdadero impacto de estos formatos sobre sus comportamientos financieros.
	
	\item \textbf{Competencia potencial y aliados:} Se identifica competidores potenciales y aliados estratégicos, como podrían ser plataformas globales de finanzas personales que ofrecen visualización avanzada en los gastos y de planificación, sin embargo, su presencia dentro del mercado colombiano es bastante marginal frente a soluciones locales y las regulaciones específicas. Al mismo tiempo, asociaciones gremiales y entidades estatales impulsan marcos de banca abierta e iniciativas de inclusión financiera que logran abrir oportunidades de colaboración para emprendimientos que aporten modelos pedagógicos robustos y métricas claras que puedan impactar dentro del modelo de educación en Colombia.
\end{itemize}

En síntesis, el mercado actual de servicios financieros es competitivo y heterogéneo. Combina una fuerte concentración de servicios tradicionales junto a una sobreoferta de aplicaciones, contenidos digitales e influencers, que, aunque ayuden con el acceso a productos e información financiera, no resuelve de forma integral las desigualdades de compresión, el seguimiento de hábitos y la toma de decisiones responsables de jóvenes y personas. Es por tal panorama que la plataforma propuesta se diferencia al integrar un modelo de usuario explícito, junto a la hipercontextualización dada al contexto del usuario. Además de utilizar un motor basado en técnicas de aprendizaje automático personalizando las rutas del conocimiento y observando su comportamiento financiero. De forma que, se posiciona como un sistema de apoyo hacia la decisión y de formación continua, más que una aplicación de registros de gastos o un canal adicional para obtener contenido financiero.
\subsection*{Conceptos de Modelo de Negocio}
\addcontentsline{toc}{subsection}{Conceptos de Modelo de Negocio}

\subsubsection*{Modelo de Negocio}
\addcontentsline{toc}{subsubsection}{Modelo de Negocio}
Nuestro servicio de planificación financiera se dirige a segmentos de clientes específicos: especialmente jóvenes adultos, trabajadores informales, nómadas digitales y en general personas con escasa educación financiera, el modelo de negocio integra lo siguiente:

\begin{itemize}
	\item \textbf{Segmentos clave:} Es la población objetivo de la que nuestro modelo de negocios se basa en un conocimiento exhaustivo acerca de las necesidades específicas de la misma.
	
	\item \textbf{Propuesta de valor:} La propuesta de valor es el factor por el cual un potencial cliente decida decantarse por una u otra empresa; la finalidad de la propuesta es buscar solucionar un problema o buscar satisfacer la necesidad que posee un cliente. Las propuestas de valor, por lo tanto, son el conjunto de productos o servicios que satisfagan los requisitos dentro de un segmento de mercado en específico.
	
	\item \textbf{Canales de distribución:} Los canales de comunicación, distribución y de venta buscan lograr el contacto entre la empresa con los clientes. Estos puntos de contacto hacia el cliente desempeñan una labor fundamental para la experiencia personal del cliente.
	
	\item \textbf{Relación con los clientes:} Las empresas buscan definir que tipo de relación buscan establecer con cada segmento del mercado, para ello, lo logran a través de la captación y la relación de los clientes, su fidelización a la empresa y, en consecuencia, la estimulación de las ventas de los servicios o productos de la empresa.
	
	\item \textbf{Fuentes de ingresos:} Se refiere al flujo de caja que genera la empresa en cada distinto segmento del mercado. Para ello existen distintas formas de ingreso tales como transacciones derivadas a partir de pagos puntuales dados por los clientes y/o pagos periódicos realizados a cambio del suministro de una propuesta de valor o de servicios pos-venta de atención dirigida hacia el cliente.
	
	\item \textbf{Actividades clave:} Este tipo de actividades son las más esenciales que debe implementar una empresa para buscar su camino al éxito, estas son necesarias para lograr el objetivo de crear y ofrecer una propuesta de valor, alcanzar los mercados de destino y lograr establecer relaciones con los clientes exitosamente y de esta forma lograr percibir ganancias en la empresa. Las actividades que se necesiten llevar a cabo varían dependiendo de la función del modelo de negocio.
	
	\item \textbf{Recursos clave:} Los modelos de negocio necesitan de recursos claves los que permiten a las diferentes empresas crear y lograr ofrecer propuestas de valor, llegar a los mercados, y conseguir establecer relaciones con segmentos de mercado y conseguir ingresos de forma similar a las actividades clave. Los recursos clave pueden ser físicos, económicos, intelectuales o humanos. Adicionalmente, la empresa puede tenerlos en su propiedad, alquilirlos u obtenerlos de otra forma gracias a sus socios clave. 
	
	\item \textbf{Socios clave:} Son las distintas alianzas que crean las empresas con otra empresas o personas con la meta de optimizar sus modelos de negocios, reducir los riesgos implicados y adquirir más recursos.
	
	\item \textbf{Estructura de costos:} La estructura de costos busca describir los principales costos de los que se incurre a la hora de trabajar dentro de un modelo de negocio determinado. La creación y entrega de valor así como el mantenimiento de las relaciones que se tiene con los múltiples clientes y la generación de las ganancias tienen un coste percibido.
\end{itemize}

\subsubsection*{Modelo de Canvas}
\addcontentsline{toc}{subsubsection}{Modelo de Canvas}
El Business Model Canvas (también conocido como el diagrama CANVA) es una herramienta de gestión estratégica desarrollada por Alexander Osterwalder e Yves Pigneur la cual tiene como objetivo permitir estructurar de forma visual los elementos clave de un negocio, como la propuesta de valor, los clientes, la infraestructura y las finanzas. Diversos estudios resaltan que incluir un modelo de negocio en un plan empresarial es fundamental para validar y fortalecer la idea antes de su implementación.

El modelo Canvas permite ser lo suficientemente sencilla para aplicarse a cualquier escenario de empresa, ya sean pequeñas, medianas o inclusive grandes empresas, independientemente de la forma de su estrategia de negocio y público objetivo. A continuación, mediante la siguiente figura observamos la estructura del modelo Canva en la cual se expone la forma en que se organiza el modelo de negocio:

\begin{figure}[H]
	\centering
	\includegraphics[width=15cm]{Imagenes/PlantillaModeloCanvas.jpg}
	\caption[{Ejemplo de modelo CANVAS.}]{\centering Ejemplo de modelo CANVAS. \textit{Fuente:} Sitio Web: (¿Cómo hacer un modelo de negocios canvas?, Redacción CEUPE, 2022)} 
	\label{fig:planmodelocanvas}
\end{figure}

En este marco, nuestro servicio dirigido a personas sin conocimientos financieros en Bogotá– surge como una propuesta innovadora que ofrece herramientas didácticas y dinámicas para la planificación financiera personal. La iniciativa se complementa con un enfoque de gamificación ligera y la construcción de una comunidad de educación financiera, buscando motivar el aprendizaje, promover hábitos saludables y fomentar la inclusión financiera.
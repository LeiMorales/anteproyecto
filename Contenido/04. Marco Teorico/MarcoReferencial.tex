\subsection*{Marco referencial}
\addcontentsline{toc}{subsection}{Marco referencial}
El proyecto se enmarca en la literatura y las políticas actuales de \textbf{inclusión financiera y educación económica}. A través del planteamiento del problema y su justificación, pudimos señalar que la alfabetización financiera es baja entre la población colombiana, lo cual motiva la necesidad de una plataforma de aprendizaje financiero. Estas cifras evidencian brechas de conocimiento económico que justifican la implementación de los servicios educativos.

Según la \textbf{OCDE} y organismos internacionales fomentan la educación financiera como instrumento de desarrollo. Colombia es miembro de la OCDE y participa activamente en eventos como la \textit{Global Money Week}, impulsando programas educativos desde la infancia. Además, en 2022 se promulgó el \textit{Decreto 1297} y otras normativas de \textit{Open Banking} que facilitan la oferta de productos centrados en el usuario \parencite{Decreto1297}. Según la OCDE, estas reformas buscan \textit{``mejorar la inclusión financiera''} y alentar la competencia mediante APIs abiertas y sandboxes (por ejemplo, InnovaSFC). En este marco, la empresa proyectada se alinea con tendencias globales: ofrecerá un servicio innovador apoyado en tecnología de punta para empoderar a usuarios inexpertos.

La \textbf{teoría económica} relevante incluye conceptos de finanzas conductuales y ciclos de vida del consumidor. Sabemos que 66 entidades de las 141 con iniciativas de educación financiera en Colombia aplican ``nudges'' conductuales (recordatorios de ahorro, mensajes positivos) para fomentar buenos hábitos \parencite{ColombiaFintech01}. Nuestro enfoque puede incorporar dichas estrategias (por ejemplo, notificaciones de ahorros regulares). Además, la Ley 1735 de 2014 (ley de inclusión financiera) de 2014 menciona en su artículo 8° ``aprovechen la tecnología disponible para la prestación de los mismos'' a servicios financieros, por tanto, existe una base legal y social que apoya iniciativas de este tipo \parencite{Ley1735}.

A nivel local, la \textbf{bibliografía académica} y de política pública ha resaltado la relación entre educación y bienestar financiero. Estudios de Asobancaria documentan que un mayor nivel de educación financiera juega un papel fundamental en la gestión de finanzas personales; sin embargo, la inclusión financiera no implica automáticamente un adecuado manejo del crédito ni mayores niveles de ahorro \parencite[p. 1, 8-9]{Asobancaria02}. Por ello, el marco referencial incorpora tanto datos estadísticos (DANE, Banco de la República y Asobancaria por nombrar algunos) como enfoques conceptuales: la importancia de la planificación familiar (p. ej. presupuesto y ahorros a largo plazo) en economías con alta informalidad y vulnerabilidad socioeconómica.

Finalmente, actores sectoriales como las entidades de supervisión y gremios del sector han llamado la atención sobre los riesgos de sobreendeudamiento, según un estudio hecho por Bravo, empresa especializada en liquidación de deudas, cerca del 43,3\% de los colombianos mantienen de tres a cinco obligaciones en mora \parencite{Infobae01}. En consecuencia, la plataforma proyectada se fundamenta en evidencia empírica y en mejores prácticas, combinando la asesoría personalizada con elementos pedagógicos para ayudar a cerrar las brechas detectadas en los estudios citados.

En suma, el marco referencial del negocio toma en cuenta que \textit{``la educación financiera desde la infancia facilita la adopción de hábitos responsables''}, y que un mayor conocimiento personal de finanzas impacta positivamente en el bienestar socioeconómico. La iniciativa propuesta busca alinearse con esas políticas públicas y respalda la visión de la OCDE según la cual la \textit{toma de decisiones informadas es clave para el bienestar financiero}.